\documentclass[sn-apa]{sn-jnl} % APA Reference Style 

\usepackage{graphicx}
\usepackage{amsmath,amssymb}

\begin{document}

\title[Supplementary Materials]{Supplementary Materials}

\maketitle

\section{Results}

\subsection{Interaction Analysis for Acoustic Cue Integration}

This additional analysis as shown in the right side of figure \ref{fig:analysis_3_plot} of the main paper, we investigated how different acoustic cues influence eye-fixations during word recognition. To do this, we built four mixed-effects models, two for each stress type at syllables 1 and 2, each with identical model structures. For both antepenultimate and penultimate models, target looks were used as the dependent variable with the four scaled measurements of the auditory stimuli (syllable-pitch, syllable-amplitude, syllable-spectral tilt, and syllable-duration) and their interactions. Additionally, \textit{word} and \textit{participant} were given random intercepts. Models with participant randome intercepts did not converge so we simplified to word only random intercepts. Following this, all models started with maximal models and were reduced using a backward stepwise selection procedure. 

The backward stepwise selection procedure began with the full model, which included all main effects and their interactions. Non-significant items were removed starting with the highest-order interactions, based on the significance levels from Wald z-tests. Each step involved refitting the model without the least significant main effect or interaction and comparing the fit of the reduced model to the previous model using AIC. The process continued until the removal of any additional terms resulted in a significant increase in AIC, indicating a poorer fit to the data. This method ensures that the final models were parsimonious, retaining only those predictors that contribute significantly to explaining the variance in target looks. While \cite{Sulpizio_McQueen_2012} confirmed analyses with reverse regression, their primary analysis focused on correlations between variables and eye-fixations. \hl{We also did a secondary analysis that includes interactions between speech cues that can be found in our online work\_flow on OSF, which was designed to allow us to capture more subtle variations across target fixations for Italian word stress. Results can be found in right figure \ref{fig:analysis_3_plot}


For the antepenultimate syllable 1 model, higher pitch led to fewer looks to the target ($\beta = -0.21$, \textit{SE} = 0.07, \textit{z} = -3.01, $p = 0.003$). Additionally, two interactions were found. First, when pitch and amplitude were both high, more looks were directed at the target ($\beta = 0.20$, \textit{SE} = 0.08, \textit{z} = 2.39, $p = 0.02$). Similarly, when both pitch and duration were higher, there were more looks to the target ($\beta = 0.20$, \textit{SE} = 0.07, \textit{z} = 2.62, $p = 0.009$).

In the antepenultimate syllable 2 model, both higher pitch ($\beta = -2.09$, \textit{SE} = 0.60, \textit{z} = -3.48, $p < 0.001$) and longer duration ($\beta = -1.25$, \textit{SE} = 0.48, \textit{z} = -2.58, $p < 0.01$) led to fewer looks at the target. When these two cues (higher pitch and longer duration) were combined, even fewer looks were directed at the target ($\beta = -1.60$, \textit{SE} = 0.61, \textit{z} = -2.62, $p = 0.009$).

For the penultimate syllable 1 model, no significant main effects were found. However, an interaction between spectral tilt and pitch ($\beta = -0.26$, \textit{SE} = 0.10, \textit{z} = -2.61, $p = 0.009$) suggested that higher tilt and pitch together led to fewer looks at the target. Additionally, when pitch and duration were both higher, more looks were directed at the target ($\beta = 0.20$, \textit{SE} = 0.06, \textit{z} = 3.13, $p = 0.002$).

In the penultimate syllable 2 model, higher amplitude led to more looks at the target ($\beta = 0.66$, \textit{SE} = 0.21, \textit{z} = 3.13, $p = 0.002$), while longer duration led to fewer looks at the target ($\beta = -0.30$, \textit{SE} = 0.12, \textit{z} = -2.53, $p = 0.01$). Several interactions were found: when amplitude and pitch were both high, there were fewer looks at the target ($\beta = -0.35$, \textit{SE} = 0.14, \textit{z} = -2.50, $p = 0.01$), and similarly, when amplitude and duration were both high, fewer looks were directed at the target ($\beta = -0.47$, \textit{SE} = 0.16, \textit{z} = -2.96, $p = 0.003$). However, when spectral tilt and pitch were both high, more looks were directed at the target ($\beta = 0.20$, \textit{SE} = 0.09, \textit{z} = 2.35, $p = 0.02$), and when pitch and duration were both high, more looks were directed at the target ($\beta = 0.26$, \textit{SE} = 0.09, \textit{z} = 2.99, $p = 0.003$).

\end{document}
