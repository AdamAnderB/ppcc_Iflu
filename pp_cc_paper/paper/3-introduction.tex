
Individual differences in language prediction have become a focal point of psycholinguistic research, revealing substantial variability in how people anticipate upcoming linguistic input. These differences are shaped by a variety of cognitive and linguistic factors, providing insights into the underlying mechanisms of language processing. Although existing studies have identified several modulators of prediction, such as working memory capacity \cite{Huettig2016,Li2023}, the majority of this research has been conducted on relatively homogeneous university student populations. This narrow focus may obscure additional factors that influence predictions outside the university WEIRD populations.

In this study, we aim to address this gap by investigating the role of individual differences in language prediction using a more heterogeneous sample. We replicate and extend the work of \citep{Sulpizio_McQueen_2012}, who explored the effects of lexical stress on word recognition, by incorporating a broader range of individual difference measures, including working memory, lexical proficiency, autism spectrum characteristics, and speech cue sensitivity (i.e., pitch, format. duration, and risetime). Our approach employs modern statistical techniques such as Generalized Linear Models (GLMs) and their mixed effect counterparts (GLMERS), LASSO regression, and k-Nearest Neighbors (k-NN) to model these effects more effectively and interpretably.

In the process of word recognition, listeners use incoming acoustic-phonetic cues to disambiguate between potential competitors. For example, when an English speaker hears the /m3/ they will activate both metal and mellon. However, as the second syllable begins clearer then the listeners will begin to disambiguate the two. While less studied, this competition also occurs on the suprasegmental level. For example, Mandarin speakers use tonal information to distinguish between the four tones of Mandarin: /ma1/ and /ma2/ are segmentally identical but tonally contrastive. 
In the case of Mandarin, the primary cue of this suprasegmental information is pitch variation. The relationship between tones and f0 contours is fairly reliable. However, other suprasegmental information like stress is less straightforward. Unlike tone, stress is less well understood across languages. Some languages use primarily duration for stress, while others primarily use pitch or spectral tilt. While less understood, stress has also been found to be used predicatively in the process of word recognition. Dutch stress   

However, the exact manner that listeners integrate such knowledge is still up for dedate. For example, do listeners \citep{Best_1995}


In this study we replicate experiment 1 from  in order better understand the contribution of individual differences in word recognition. 