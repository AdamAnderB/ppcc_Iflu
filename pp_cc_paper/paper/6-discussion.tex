
The present study aimed to replicate and extend the findings of \cite{Sulpizio_McQueen_2012} on the role of lexical stress in word recognition among Italian speakers. Through our replication and extension, we sought to provide a deeper understanding of individual differences in processing Italian lexical stress, utilizing a more heterogeneous sample. Additionally, we aim to explore the relationship between cognitive measures and speech cue integration during Italian word recognition of suprasegmentally contrastive competitors.

\subsection{Replication findings}
Like \cite{Sulpizio_McQueen_2012}, our findings suggest that there is no differences between looks to targets over the time course of word recognition for penultimate or antepenultimate tri-syllabic words. That is, for both stress types, looks to targets increase over the time course of word recognition. One difference between our results and \cite{Sulpizio_McQueen_2012} is in which cues are primarily used during Italian stress processing.

\subsubsection{Acoustic metric comparisons}
In contrast to \cite{Corretta2023} findings, measures are all reasonably close between studies with the exception of the scale for spectral tilt being different. However, all measurements are similar and t-tests reveal similar magnitude and identical direction of differences between all comparisons. Due to directionality and magnitude similarities for spectral tilt measure comparisons, the difference in spectral tilt scales is likely due to differences in calculating spectral tilt rather than segmentation differences. The method used in our study comes directly from \citep{sluijter1996spectral,cutler2007dutch}. However, upon examination, the exact manner that spectral tilt is calculated can vary from study to study. Our approach is available on OSF in the text\_grid\_extractor.Rmd. Additionally, the variation between different acoustic cues is not consistent across stress. For example, the largest and most apparent difference in acoustic cues is found in duration (see Figure \ref{fig:raw_acoustics}) with an interaction between first and second syllable and duration length. The difference between amplitude and pitch both go down between syllable 1 and 2 across stress types with antepenultimate words having a greater decrease. Spectral tilt, on average, is higher in both stress types for stressed syllables. That is, acoustically, duration is the most salient and reliable cue to differentiate antepenultimate and penultimate words, followed by pitch, amplitude, and spectral tilt, respectively.


\subsubsection{Target and competitor}

Moreover, we found no significant effect of stress on fixation patterns, aligning with \cite{Sulpizio_McQueen_2012}'s conclusion that stress does not significantly influence word recognition within the tested tri-syllabic Italian words. This consistency strengthens the evidence that, while suprasegmental features like stress are integrated during word recognition, they may not independently alter the fixation dynamics overall across different words for all participants. That is, looks to targets could be averaging out to be similar across both stress types even if differences are occurring\ for words with specific acoustic cue characteristics, which we explore below. 

\subsubsection{Target bias}

The second analysis (bias analysis) did not find differences between stress types for target fixations. That is, participants did not look more toward penultimate words even though they are more common in the Italian lexicon. 

\subsubsection{Cue integration}

Third, in our cue integration analyses of first syllables, like \cite{Sulpizio_McQueen_2012}, we did not find any main effects for acoustic cues in first syllable penultimate word target fixations. However, we did find an interaction between syllable-spectral tilt and syllable-pitch. This may be because words with higher spectral tilt and pitch during the first syllable may be perceived as stressed due to large overlap with antepenultimate stress cues (see Figure \ref{fig:raw_acoustics}). The interaction between pitch and duration, contrastively, leads to more target fixation, which may be due to duration being a primary cue of stress in Italian \cite{Alfano2006}. On the other hand, for the antepenultimate stress words at the first syllable, higher pitch first syllables seem to cause confusion and make participants look away from targets (possibly towards competitors). This could be due to the participants perceiving the higher pitch as stress. This notably contrasts with \cite{Sulpizio_McQueen_2012}'s finding that amplitude is a predictor during the first syllables of antepenultimate words, which we did not find. The interactions between syllable-pitch and syllable-amplitude as well as syllable-pitch and syllable-duration also indicate complex cue integration supporting the use of multiple competing cues simultaneously \citep{Kong2016}.

In terms of second syllables, we found less consistent results with \cite{Sulpizio_McQueen_2012}. We found a main effect of amplitude for penultimate second syllables, where they found no significant cues in penultimate words. That is, participants use amplitude in the second syllable to determine if a word has penultimate stress (penultimate words are not the default but rather recognized during the second syllable). For the antepenultimate second syllables, higher pitch and duration lead to less target looks, which suggests that participants are using pitch and duration information in the second syllable to lower competition with penultimate words during word recognition \cite{McMurray2019}. These results together suggest that participants are not only using abstract knowledge that penultimate words are more frequent, but rather using both the first and second syllable stress cue information to determine the identity of a word for both penultimate and antepenultimate words.

\subsection{Individual differences in word recognition}
Extending beyond the replication, our analysis incorporated individual difference measures such as working memory, lexical proficiency (English and Italian), AQ, and sensitivity to speech cues (pitch, formant, duration, and risetime). Our findings underscore the importance of these individual differences in lexical stress processing.

The LASSO regression and GLMM analyses highlighted significant predictors of target looks, particularly Lextale scores and auditory processing skills. For instance, higher English LexTale and LextITA scores were associated with increased looks to targets, indicating that greater lexical proficiency enhances word recognition accuracy. Additionally, the interactions between these cognitive measures and acoustic cues suggest complex, non-linear relationships that influence how listeners integrate stress information during word recognition.


\subsubsection{Target and competitor - individual differences}
First, in the target and competitor individual difference for penultimate words a formant sensitivity during the first 1.5 syllables leads to early target fixations suggesting that formant ability is a key cue for penultimate word recognition. During syllable 2, sensitivity to duration and lower pitch decreased looks to targets, meaning that lower pitch second syllables increase competition for Italian listeners. Similarly, for partiicpants with higher duration sensitivity less looks occur in the second syllable, which may be an outcome of the sheer variability in  


In combination, these effects suggest that being sensitive to durational differences creates higher competition during lexical access and that high Italian vocabulary knowledge results in less competition with antepenultimate competitors when hearing a penultimate word. Additionally, the increasing deltas found for the interaction between English lexical and first through second syllables may indicate a bilingual advantage in first language recognition for those that have higher English abilities. 

For the similar antepenultimate model, pitch d', English Lextale, and LextITA all have positive effects on word recognition during the first two syllables. The English Lextale and LextITA scores improving word recognition could again be a general word knowledge advantage of bilinguals. The benefit of pitch discrimination ability may be due to the fact that antepenultimate first syllables are generally higher than their penultimate counterparts. The additional positive relationship between the interaction between English Lextale and syllable may be due to the fact that English has more first-syllable stress words. That is, those with greater English skills are co-activating English lexical words when they hear primary stress on the first syllable.

\subsubsection{Cue integration - individual differences}
Our investigation into cue integration revealed nuanced effects of individual differences on acoustic cue integration across fixation patterns for both penultimate and antepenultimate words. Here, we discuss the results of the four cue integration individual difference analyses. We start with our first syllable penultimate stress model and continue to the first syllable antepenultimate model, ending with their second syllable counterparts.

Like \cite{Sulpizio_McQueen_2012}'s penultimate acoustic cue model, our individual difference analysis revealed no significant effects within the first syllable of penultimate words. However, The simple cue integration model without individual difference during the replication analysis found interactions between speech cue - the lack of effect here suggests that once individual difference are accounted for those interactions no longer help in the process of word recognition. 

For the antepenultimate first syllable model, pitch d', LextITA, and English Lextale all had positive relationships with target fixations. The negative significant interactions (between duration d' and word duration and between spectral tilt and risetime) are likely due to the fact that antepenultimate first syllables more often have longer duration or higher amplitude than second syllables. That is, participants are confusing antepultimate words for penultimate counterparts if cues align with penultimate patterns.

In the penultimate second syllable individual difference model, positive effects on both English Lextale and LextITA again support a bilingual advantage in the process of word recognition. The interactions in this model again point toward gradual integration that is dependent on the specific cognitive individual differences. For example, positive d' duration discrimination of an individual and a word having higher pitch during the second syllable leads to less looks at the penultimate word. This indicates that those that use duration well may confuse words that are using pitch as an indicator of stress rather than duration. This is supported also by the fact that, in contrast, those that have high formant discrimination abilities look more toward targets when the duration is longer, which could be due to there being more formant information available (i.e., longer vowels). This finding suggests variant strategies in the process of word recognition across stress in Italian lexical processing. Our findings, here, suggest that Italian listeners may not segment at the penultimate first syllable because it does not present as a strong syllable requiring acoustic validation \cite{CutlerNorris1988}. This aligns with observation that penultimate stress patterns in Italian are recognized by default, without the need for additional acoustic cues \cite{Sulpizio_McQueen_2012}.

Lastly, the antepenultimate second syllable individual difference model found a negative effect for duration d', which means that the participants with greater duration sensitivity are still considering penultimate counterparts during the second syllable even after hearing a stressed first syllable. The interaction between English Lextale and syllable-pitch, here, again supports a model of word recognition that integrates across the bilingual lexicon. It seems that for those that have more English experience, English stress cues like pitch could have a greater effect. While pitch is also a predictor of stress in Italian, it may be that the participants are simply expecting a greater amount of first syllable stressed words due to their higher English proficiency.

Finally, The application of k-nearest neighbors (k-NN) and random forest models provided further insights, confirming the significance of pitch and duration in predicting target fixations to stress patterns across both syllables. These machine learning approaches reinforced our parametric findings and demonstrated the robustness of our analytical framework. Our study contributes to the growing body of literature on individual differences in language processing by demonstrating how lexical proficiency and auditory processing skills modulate the integration of lexical stress cues. Unlike prior research, working memory did not come up significant in the parametric models. However, our k-NN and random forest analysis indicate that working memory is an important modulator during word recognition. Similarly, ASDQ scores also showed up as an important variable in these models. These results indicate a need for examining non-linear relationships between individual differences and cues during word recognition in order to fully understand how cognitive abilities affect the time course of lexical processing. The lack of findings for working memory in parametric models could also be due to the fact that we are looking at low frequency words at the level of word recognition rather than sentence processing. These findings have important implications for models of spoken word recognition, suggesting that cognitive and linguistic factors play a central role in word recognition and that various strategies are employed by different listeners for each word.

Future studies should explore a wider range of individual difference measures and employ longitudinal designs to assess how these factors interact over time. Additionally, examining other languages and specialized populations may provide a more comprehensive understanding of the universality and specificity of these effects. Additionally exploring how continuum manipulated speech contrasts affect Italian word recognition would provide additional insight into the exact cues that are most useful for word recognition.