We carried out a web-based replication and extension of \cite{Sulpizio_McQueen_2012}'s Experiment 1 with a larger and broader Italian population in order to better understand the individual differences involved in processing and predicting Italian lexical stress. We first summarize our findings given the four research questions involved in the replication. We then summarize our findings for our exploratory extension research question. We end with limitations and a brief conclusion.

\subsection{Web-based replication of Sulpizio and McQueen (2012)}
\subsubsection{The acoustic cues involved in Italian stress production are duration, amplitude, pitch, and spectral tilt}
We first carried out our own acoustic analysis of the audio stimuli used in \cite{Sulpizio_McQueen_2012} (see Table \ref{tab:acoustics}). Whereas we increased researcher degrees of freedom \citep{Corretta2023, roettger2019researcher}, we found remarkably similar results to the original study (see Figure \ref{fig:raw_acoustics}). Moreover, all of our mean measurements and t-tests revealed similar magnitude and identical direction of differences between all comparisons as reported by \cite{Sulpizio_McQueen_2012}. For both stress types, amplitude and pitch decreased from the first vowel to the second vowel. For antepenultimate stress, spectral tilt and duration both decreased from the first to the second vowel. In other words, for antepenultimate stressed words, all acoustic cues decreased from the first vowel to the second vowel. In contrast, for penultimate stressed words, spectral tilt is not a reliable cue and duration increased from the first vowel to the second vowel.

It is worth noting that our difference in spectral tilt scales is likely due to differences in calculating spectral tilt rather than segmentation differences. The method used in our study comes directly from \citep{sluijter1996spectral,cutler2007dutch}, which we have made available on our OSF page. The exact manner in which spectral tilt is calculated can vary from study to study. We echo recent calls to make phonetics research methods explicitly detailed and freely available to readers and reviewers \citep{roettger2019emergent, roettger2019researcher}. With the advent of the Open Science Framework, researchers should minimally, if possible, include scripts and step-by-step guides for all acoustic analyses. We believe this type of commitment by researchers will only improve the rigor and quality of the work.

\subsubsection{Antepenultimate and penultimate stressed Italian words are recognized with similar time courses}
Next, we found that listeners rapidly integrated segmental and suprasegmental information to recognize spoken Italian words. Targets were identified at an early stage of word recognition---somewhere during syllables 1.5 and 2---well before any disambiguating segmental information was available to listeners. Moreover, we found that stress type did not significantly influence word recognition. Competitors (and distractors) were ruled out by the second syllable, irrespective of stress type. These findings are in line with \cite{Sulpizio_McQueen_2012}'s results, as well as research on real-time Dutch stress recognition \citep{Reinisch2010}. Our results further demonstrate the success (and relative ease) of carrying out a web-based eye-tracking study with a larger, more heterogeneous population of participants \citep{Vos_2017, bramlett_wiener_24-AOW}.

\subsubsection{On the surface, Italian speakers are not biased towards penultimate stressed words}
Like \cite{Sulpizio_McQueen_2012}, we found no direct evidence to indicate that participants predict penultimate stressed words over antepenultimate stressed words despite their higher occurrence in the Italian lexicon. In our following acoustic cue analysis and individual differences analysis, however, we demonstrate this is not the case: a more careful examination involving the acoustic cues---just as \cite{Sulpizio_McQueen_2012} demonstrated---reveals complex and interesting patterns.

\subsubsection{Italian listeners use pitch cues early and duration, amplitude, and spectral tilt cues later to recognize stress}
Our last replication analysis involved examining which acoustic cues drove eye fixations in real-time. Here we went beyond \cite{Sulpizio_McQueen_2012} by including two-way interactions in our models. The second column in Table \ref{tab:cue-summary} summarizes our results. We found that pitch and its combination with other acoustic cues contributed to eye fixations during the first syllable. A higher pitch by itself is uninformative since both stress types show higher first vowel pitches, hence the reduced looks we observed for antepenultimate stressed targets. However, a higher pitch combined with a higher amplitude or a higher pitch combined with a longer duration resulted in greater looks to stressed antepenultimate targets. In both instances, this combination of cues may have been helpful in identifying antepenultimate over penultimate stressed words and suggests listeners integrated multiple acoustic cues at an early stage of word recognition \citep{Kong2016, mcmurray_2008}. Interestingly, during the first syllable, we also observed the same increase in looks to (unstressed) penultimate targets when a higher pitch was combined with a longer duration, suggesting the combination of a high pitch and long duration may have attracted listeners' attention given the relative perceptual salience of F0 and duration over intensity and spectral cues (see \cite{ip2022search, beckman2012stress} for discussions). Alternatively, this increase in looks could be interpreted as evidence for a bias towards penultimate stressed words. 

For the second syllable, we found that pitch continued to play a role in driving eye fixations, though other cues were relied on, as well. For (unstressed) antepenultimate targets, there were fewer looks given a higher pitch and a longer duration. During this second syllable, listeners would already have identified the antepenultimate target, hence the reduced looks. Moreover, a longer duration would indicate a penultimate stressed word while a higher second vowel pitch is not a cue used for either stress type. For (stressed) penultimate words, a higher pitch combined with a longer duration once again resulted in increased looks to the target, potentially because of the cues' perceptual salience \citep{ip2022search, wright2009review, beckman_1994} or because penultimate stress involves a longer second vowel duration. A longer duration alone, however, led to fewer looks. This cue alone may not be sufficient for identification (cf. \cite{Alfano2006, Alfano2009}) or at least not useful by the second syllable. Amplitude played a more curious role. A higher amplitude alone led to greater looks to the penultimate target. When a higher amplitude was combined with a higher pitch or a longer duration, however, it led to fewer looks. These findings are somewhat in conflict with \cite{Sulpizio_McQueen_2012} which claimed amplitude alone was a sufficient cue for stress identification, especially for antepenultimate recognition. We found no such evidence to support this claim. Our results are somewhat more in line with \cite{Tagliapietra2005} who argued that amplitude and duration were both used. Whereas we found evidence for both cues being used, they were not always aligned. In other words, higher amplitude and longer duration led to fewer looks to the penultimate target while a longer duration and higher pitch led to more looks. 

The differences between our results and \cite{Sulpizio_McQueen_2012}'s results can be attributed to at least three different factors. First, we used a very different population of adults we recruited online. This was a much more heterogeneous population, including older adults not in a university setting. This speaks to the importance in testing varied populations and ensuring that non-WEIRD populations are part of psycholinguistics research \citep{frost2021investigating}. Related to this, our sample size was over twice as large as the original study's. It may be that greater statistical power yielded different results. Second, we conducted the study online. Though we included attention checks, we cannot say for certain that our participants were as engaged as the original study's in-person testing (see \cite{rodd2024moving} for additional discussion). Third, our modeling approach differed from the original study's (although we only included eye-fixations that showed greater than chance face validation to mitigate possible lack of attention). As we discussed, this was done to better understand the data and allow for more nuanced findings, such as two-way interactions. Had \cite{Sulpizio_McQueen_2012} tested for interactions, their results may have been closer to those we report here.

Taken together, we do not directly replicate \cite{Sulpizio_McQueen_2012}'s acoustic cue findings. We do, however, paint a richer picture involving real-time multiple prosodic cue integration \citep{McMurray2019, mcmurray_2008} to determine the identity of a word for both antepenultimate and penultimate words. This information appears to be integrated with abstract knowledge of stress patterns' frequency of occurrence. We further unpack this complex process by now turning to our exploratory individual differences study.

% Please add the following required packages to your document preamble:
% \usepackage{graphicx}
\begin{table}[ht]
\centering
\caption{Summary of replication and extension acoustic cue analyses by stress type and syllable}
\label{tab:cue-summary}
\begin{tabular*}{\textwidth}{@{\extracolsep{\fill}} l p{6cm} p{6cm}}
\hline
\textbf{} &
  \makecell{\textbf{Our replication}\\ \textbf{results}} &
  \makecell{\textbf{Our individual}\\ \textbf{differences}\\ \textbf{extension results}} \\ \hline
\multicolumn{3}{@{\extracolsep{\fill}} l}{\textbf{Antepenultimate}} \\ \hline
\textbf{\makecell{Syllable 1\\ Stressed}} &
  \begin{tabular}[t]{@{}p{5cm}@{}}-- Fewer looks given higher pitch\\ -- Greater looks given higher pitch and higher amplitude\\ -- Greater looks given higher pitch and longer duration\end{tabular} &
  \begin{tabular}[t]{@{}p{5cm}@{}}-- Greater looks given higher pitch d'\\ -- Greater looks given higher English LexTALE\\ -- Fewer looks given longer duration and higher duration d'\\ -- Fewer looks given higher spectral tilt and higher risetime d'\end{tabular} \\ \hline
\textbf{\makecell{Syllable 2}} &
  \begin{tabular}[t]{@{}p{5cm}@{}}-- Fewer looks given higher pitch\\ -- Fewer looks given longer duration\end{tabular} &
  \begin{tabular}[t]{@{}p{5cm}@{}}-- Fewer looks given higher pitch and higher English LexTALE\end{tabular} \\ \hline
\multicolumn{3}{@{\extracolsep{\fill}} l}{\textbf{Penultimate}} \\ \hline
\textbf{\makecell{Syllable 1}} &
  \begin{tabular}[t]{@{}p{5cm}@{}}-- Fewer looks given higher pitch and higher spectral tilt \\ -- Greater looks given higher pitch and longer duration\end{tabular} &
  \begin{tabular}[t]{@{}p{6cm}@{}}No effects found\end{tabular} \\ \hline
\textbf{\makecell{Syllable 2\\ Stressed}} &
  \begin{tabular}[t]{@{}p{5cm}@{}}-- Greater looks given higher amplitude\\ -- Fewer looks given longer duration\\ -- Fewer looks given higher pitch and higher amplitude \\ -- Fewer looks given higher amplitude and longer duration\\ -- Greater looks given higher pitch and higher spectral tilt\\ -- Greater looks given higher pitch and longer duration\end{tabular} &
  \begin{tabular}[t]{@{}p{5cm}@{}}-- Fewer looks given longer duration\\ -- Greater looks given higher Italian LexITA\\ -- Greater looks given longer duration and higher formant d'\end{tabular} \\ \hline
\end{tabular*}
\end{table}

\subsection{Individual differences in Italian word recognition}
\subsubsection{Language proficiency in the L1 and L2 affect Italian word prediction}

In our exploratory analysis, we found a clear asymmetry in our results driven by L1 and L2 lexical knowledge. Participants with higher L1 Italian LexITA proficiency looked to the (stressed) penultimate second syllable at a greater rate. This supports the claim that listeners store abstract knowledge about Italian stress patterns \citep{Sulpizio_McQueen_2012}. We add to this finding by demonstrating that this knowledge varies across individuals. In other words, individuals with a higher Italian proficiency appear to make use of such abstract knowledge greater than those with reduced Italian proficiency. This finding is in line with reading research which has shown larger lexical knowledge results in greater statistical knowledge \citep{mirman2008attractor, kuperman2013reassessing}. In contrast, participants with higher L2 English LexTALE proficiency looked to the (stressed) antepenultimate first syllable at a greater rate. We interpret this pattern as evidence of the L2 affecting L1 \citep{marian2003competing, dijkstra2002architecture}. English may be helpful in recognizing antepenultimate word recognition due to the higher frequency of first syllable stress in English \cite{cutler2007dutch}. That is, those with greater English skills are co-activating more first syllable stressed words. Surprisingly, we did not find evidence that there were any negative effects of higher language proficiency on word recognition. This is in contrast with \cite{primativo2013bilingual}'s reading study. However, we did find that during the second syllable of antepenultimate words, higher pitch and higher LexTALE predict less target fixations, which may suggest an interaction between English proficiency and cue use. That is, if you are more proficient in English, high pitch antepenultimate second syllables are less expected. One explanation for these conflicting results is that while production may be somewhat hindered by cross-linguistic differences in stress cues, perception benefits from it.

\subsubsection{Working memory and autism spectrum quotient do not predict eye movements}
To our surprise, we found null effects of both working memory and autism spectrum quotient in all our exploratory models. This was unexpected given the considerable number of studies that have found working memory as a reliable predictor \citep{Traxler2009, Huettig2016}. Why might this be? Both measures appeared to be normally distributed (see Figure \ref{fig:plot_raw_task}), indicating we had a wide range of participants and behavior. Both tasks were fairly reliable as measured by Cronbach's alpha. These null results may be due to the fact that we also took into account acoustic cue sensitivity and language proficiency. The effects found in other studies for prosody deficiency for autism characteristics may be the result of not controlling for more basic cognitive measures \cite{grossman2023relationship, Liu2018}. Similarly, it could also be that either variability in working memory is not as useful for prosodic processing over only two syllables (floor effect) or that our sample was still not varied enough in working memory to capture differences. Similarly, the fairly straightforward design of our study may have been too simple to observe robust effects of autism spectrum quotient (cf. \cite{Sinagra2022} which tested prosody detecting involving face masks that obscured facial cues).

\subsubsection{Auditory sensitivity contributed to variable eye movements}
 Our exploratory study highlighted how stress cues are used variably across individuals. Our results suggest variable processing strategies that not only depend on the cue sensitivity of the individual but also on the specific multi-dimensional cues for each word. Our analyses showed that pitch sensitivity plays a role in first syllable antepenultimate stress detection, which aligns with our earlier finding that pitch is a central cue in predicting eye-fixations for stress. If pitch is as critical as our replication suggested, pitch sensitivity would naturally be important for stress detection. This is indeed what we found.

In addition to pitch sensitivity, formant sensitivity positively affected penultimate stress detection, which speaks to the relative inseparability of stress from segmental information in Italian \citep{CutlerNorris1988, Tagliapietra2005}. Our by-syllable analysis revealed that formant sensitivity was particularly useful for penultimate words with longer second syllable durations. Words that have longer vowels have more formant information, suggesting that the usefulness of formant sensitivity is modulated by their being sufficient formant information.

For duration, less duration sensitivity led to fewer looks to penultimate targets during the second syllable. Though interestingly, longer duration combined with higher formant sensitivity predicted more target fixations. targets across antepenultimate words overall. The negative relationship between duration sensitivity and target looks appears counter-intuitive. However, in the case of duration sensitivity during the first syllable of antepenultimate words, a negative significant interaction (between duration d$'$ and word duration) is likely due to the fact that antepenultimate first syllables often have longer duration than second syllables. That is, for participants with low duration sensitivity, short-duration items do not cause them to look away from the targets even though the short-duration syllable sounds more like the competitor, i.e., more unstressed. This same phenomenon occurs with risetime sensitivity and spectral tilt of a word. When participants with low risetime sensitivity perceive a word with low spectral tilt they do not look away from targets even though the cues do not align with the target. 

We take this finding as evidence of variable processing strategies. Specifically, this result corroborates theoretical and computational predictions from \cite{mcmurray_2009}, which suggests more gradient processing for those with higher cue sensitivity. In other words, a low sensitivity to a cue results in more categorical processing. Low duration sensitivity participants were looking at the target more than high sensitivity individuals for short first syllables (when the acoustic cue was mismatching the target). \cite{mcmurray_2009}'s modeling suggests that individuals with higher sensitivity are better at using detailed, gradient information to distinguish phonetic contrasts, while those with lower sensitivity might rely more on categorical distinctions, leading to less accurate recognition when cues are mismatched. In our case, individuals with reduced sensitivity to a cue appeared to process the cue more categorically and less gradiently. What is perhaps most interesting is how gradient processing interacted with listeners abstract knowledge of Italian stress. It appears that the penultimate stress was considered to be the 'default' by those with lower cue sensitivity  as suggested by \cite{Sulpizio_McQueen_2012}. However, for those with greater cue sensitivity, abstract knowledge plays a lesser role as they use acoustic cues to a greater extent in Italian word stress recognition. 

\subsubsection{Limitations}

Despite a larger and more diverse sample size than \cite{Sulpizio_McQueen_2012}, our participant pool was still limited to Italian speakers recruited online, potentially introducing selection bias. Conducting the study online also presented inherent challenges in controlling the experimental environment. Although we implemented attention checks, the variability in participants' home environments, such as noise levels and device quality, might have affected the results (see \cite{bramlett_wiener_24-AOW} for discussion).

Our statistical analyses, while providing nuanced insights and controlling for colinearity, introduced complexity that complicates direct comparisons with the original study. The inclusion of two-way interactions and LASSO regression might have led to divergent findings. Additionally, our findings on individual differences, particularly the null effects for working memory and autism spectrum quotient, suggest an interconnected relationship between cue sensitivity, language proficiency (in the L1 and L2), working memory, and AQ. Our online R code shares further exploratory correlation analyses for interested readers. Future research should consider a broader array of individual difference measures, including cognitive flexibility and attention control, to provide a more comprehensive picture.

Furthermore, focusing exclusively on Italian limits the generalizability of our findings to other languages. Cross-linguistic studies are needed to determine the universality of our findings, as stress patterns and their acoustic correlates vary significantly across languages \cite{cutler2007dutch}. The cross-sectional nature of our study also precludes conclusions about the developmental trajectory of stress processing abilities. Longitudinal studies tracking individuals over time could shed light on the stability and consistent use of stress cues for individuals in different environments. Future studies should explore a wider range of individual difference measures and employ longitudinal designs to assess how these factors interact over time. Examining other languages and specialized populations, as well as exploring how continuum-manipulated speech contrasts affect Italian word recognition, would provide additional insights into the cues most useful for word recognition.

%Finally, The application of k-nearest neighbors (k-NN) and random forest models provided further insights, confirming the significance of pitch and duration in predicting target fixations to stress patterns across both syllables. These machine learning approaches reinforced our parametric findings and demonstrated the robustness of our analytical framework. Our study contributes to the growing body of literature on individual differences in language processing by demonstrating how lexical proficiency and auditory processing skills modulate the integration of lexical stress cues. Unlike prior research, working memory did not come up significant in the parametric models. However, our k-NN and random forest analysis indicate that working memory is an important modulator during word recognition. Similarly, ASDQ scores also showed up as an important variable in these models. These results indicate a need for examining non-linear relationships between individual differences and cues during word recognition in order to fully understand how cognitive abilities affect the time course of lexical processing. The lack of findings for working memory in parametric models could also be due to the fact that we are looking at low frequency words at the level of word recognition rather than sentence processing. These findings have important implications for models of spoken word recognition, suggesting that cognitive and linguistic factors play a central role in word recognition and that various strategies are employed by different listeners for each word.

