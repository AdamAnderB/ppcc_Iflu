We carried out a web-based replication and extension of \cite{Sulpizio_McQueen_2012}'s Experiment 1 with a larger and broader Italian population in order to better understand the individual differences involved in processing and predicting based on Italian lexical stress. We first summarize our findings given the four research questions involved in the replication. We then summarize our findings for our exploratory extension research question. \hl{We end with comments on replications, our study's limitations, and a brief conclusion.}

\subsection{Web-based replication of \cite{Sulpizio_McQueen_2012}}
\subsubsection{The acoustic cues involved in Italian stress production are duration, amplitude, pitch, and spectral tilt}
We first carried out our own acoustic analysis of the audio stimuli used in \cite{Sulpizio_McQueen_2012} (see Table \ref{tab:acoustics}). Whereas we increased researcher degrees of freedom \citep{Corretta2023, roettger2019researcher} by carrying out our own acoustic tagging and measuring of the stimuli, we found remarkably similar results to the original study (see Figure \ref{fig:raw_acoustics}). Moreover, all of our mean measurements and t-tests revealed similar magnitude and identical direction of differences between all comparisons as reported by \cite{Sulpizio_McQueen_2012}. For both stress types, amplitude and pitch decreased from the first vowel to the second vowel. For antepenultimate stress, spectral tilt and duration both decreased from the first to the second vowel. In other words, for antepenultimate stressed words, all acoustic cues decreased from the first vowel to the second vowel. In contrast, for penultimate stressed words, spectral tilt is not a reliable cue and duration increased from the first vowel to the second vowel.

It is worth noting that our difference in spectral tilt scales is likely due to differences in calculating spectral tilt rather than segmentation differences. The method used in our study comes directly from \cite{sluijter1996spectral,cutler2007dutch}, which we have made available on our OSF page. The exact manner in which spectral tilt is calculated can vary from study to study. We echo recent calls to make phonetics research methods explicitly detailed and freely available to readers and reviewers \citep{roettger2019emergent, roettger2019researcher}. With the advent of the Open Science Framework, researchers should, if possible, include scripts and step-by-step guides for all acoustic analyses. We believe this type of commitment by researchers will only improve the rigor and quality of the work.

\subsubsection{Antepenultimate and penultimate stressed Italian words are recognized with similar time courses}
Next, we found that listeners rapidly integrated segmental and suprasegmental information to recognize spoken Italian words. Targets were identified at an early stage of word recognition---somewhere during syllables 1.5 and 2---well before any disambiguating third syllable segmental information was available to listeners. Moreover, we found that stress type did not significantly influence word recognition and competition. Competitors (and distractors) were ruled out by the second syllable, irrespective of stress type. These eye-tracking findings are in line with \cite{Sulpizio_McQueen_2012}'s results, as well as research on real-time Dutch stress recognition \citep{Reinisch2010}, Mandarin and Cantonese tone recognition \citep{zou_2022, qin_2022, Nixon2016}, and Japanese pitch accent recognition \citep{Ito2024}. 




\subsubsection{Italian speakers are not biased towards penultimate stressed words}
Like \cite{Sulpizio_McQueen_2012}, we found no direct evidence of eye-fixations to indicate that participants predict penultimate stressed words over antepenultimate stressed words despite their higher occurrence in the Italian lexicon. \hl{If we do not consider the incoming acoustics, listeners are not biased towards one stress pattern over another. These results, now consistent across two populations of speakers, imply that the resting activation of these stress pairs is roughly similar despite penultimate stressed words being more frequent in speech. Any differences in eye-movements must thus represent real-time updating of the likely word given the presence or absence of acoustic cues.}

\subsubsection{First syllable spectral tilt and pitch cues predict recognition of antepenultimate stressed words}
We found that during the first syllable, only looks to the antepenultimate stressed words were affected by the incoming acoustics (see second column in Table \ref{tab:cue-summary}). A higher spectral tilt---thus indicating a more likely drop in the second syllable---resulted in greater looks to the antepenultimate stressed word. Spectral tilt decrease is a reliable cue for antepenultimate stressed words but not for penultimate stressed words. This pattern of behavior aligns with our acoustic results and replicates what \citep{Sulpizio_McQueen_2012} initially found in their study. 

We also found that a higher pitch resulted in fewer looks to the antepenultimate stressed word. This is a new finding not reported in \cite{Sulpizio_McQueen_2012} or, as far as we know, other research. \hl{We interpret this result as evidence that f0 height on the first syllable may be important information for identifying stress, or at least it was for the stimuli we used.} Figure \ref{fig:raw_acoustics} \hl{shows in the top row, second column how speakers generally produced penultimate words with a higher f0 in both syllables as compared to antepenultimate words. Listeners may have been associating a higher f0 with penultimate stressed words, and therefore looked less to the antepenultimate stressed word upon hearing f0 information during the first syllable.}

Like \cite{Sulpizio_McQueen_2012}, we found that looks to penultimate words were not affected by incoming acoustic information. \hl{If we assume that antepenultimate stressed words are perceived first as a result of stress being carried on the first syllable, then the activation of these words should increase given supporting acoustic cues (like a higher spectral tilt). Similarly, the activation should decrease given conflicting acoustic cues (like a higher f0). This is in-line with our understanding of real-time lexical competition and cue integration} \citep{Marslen1980, Kong2016, mcmurray_2008}.

We did not find evidence that listeners made use of amplitude cues. This is in contrast with \cite{Sulpizio_McQueen_2012} who found that the decrease in amplitude from the first to the second syllable led to greater looks to the antepenultimate target. Their finding was somewhat unexpected given both antepenultimate and penultimate stress patterns show similar amplitude decreases from syllable 1 to syllable 2. Amplitude is also less informative than a cue like spectral tilt, which is sufficient for stress pattern discrimination. The differences between our results and \cite{Sulpizio_McQueen_2012}'s results can be attributed to at least three different factors. 

First, we used a very different population of adults we recruited online. This was a much more heterogeneous population of adults. This speaks to the importance in testing varied populations and ensuring that non-WEIRD populations are part of psycholinguistics research \citep{frost2021investigating}. Second, we conducted the study online. Though we included attention checks, we cannot say for certain that our participants were as engaged as the original study's in-person testing (see \cite{rodd2024moving} for additional discussion). Third, our modeling approach differed considerably from the original study's (although we only included eye-fixations that showed greater than chance face validation to mitigate possible lack of attention). It may also be that our greater statistical power yielded different results. 

Thus, our evidence suggests that spectral tilt and f0 are useful acoustic cues for stress identification on the first syllable. Neither amplitude nor duration were found to be reliable predictors in our models (cf. \cite{Alfano2006, Alfano2009, Tagliapietra2005}). Because these findings were unexpected given previous research, we carried out an exploratory interaction model (see Figure \ref{fig:analysis_3_plot}. These results suggest a more complex time course of cue integration (see supplemental materials for further discussion) and involve duration and amplitude, which may better reflect previous finding \cite{Alfano2006, Alfano2009, Tagliapietra2005}. A more proper hypothesis-driven study is needed to make sense of these exploratory results. We next turn to the individual differences findings.



% Please add the following required packages to your document preamble:
% \usepackage{graphicx}
\begin{table}[ht]
\centering
\caption{Summary of replication and extension acoustic cue analyses by stress type and syllable}
\label{tab:cue-summary}
\begin{tabular*}{\textwidth}{@{\extracolsep{\fill}} l p{6cm} p{6cm}}
\hline
\textbf{} &
  \makecell{\textbf{Our replication}\\ \textbf{results}} &
  \makecell{\textbf{Our individual}\\ \textbf{differences}\\ \textbf{extension results}} \\ \hline
\multicolumn{3}{@{\extracolsep{\fill}} l}{\textbf{Antepenultimate}} \\ \hline
\textbf{\makecell{Syllable 1\\ Stressed}} &
  \begin{tabular}[t]{@{}p{5cm}@{}}-- Greater looks to words with higher spectral tilt\\ \end{tabular} &
  \begin{tabular}[t]{@{}p{5cm}@{}}-- Fewer looks to words with higher spectral tilt given lower risetime d$'$ sensitivity\\ -- Greater looks given higher pitch d$'$ sensitivity\\ -- Fewer looks to words with longer duration given lower duration d$'$ sensitivity\end{tabular} \\ \hline
\textbf{\makecell{Syllable 2}} &
  \begin{tabular}[t]{@{}p{5cm}@{}}-- Fewer looks given higher pitch\\\end{tabular} &
  \begin{tabular}[t]{@{}p{5cm}@{}}-- Fewer looks given higher pitch \\--Greater looks given higher English LexTALE  \\--Fewer looks to words with higher pitch given higher risetime d$'$ sensitivity\end{tabular} \\ \hline
\multicolumn{3}{@{\extracolsep{\fill}} l}{\textbf{Penultimate}} \\ \hline
\textbf{\makecell{Syllable 1}} &
  \begin{tabular}[t]{@{}p{5cm}@{}}--No effects founds\end{tabular} &
  \begin{tabular}[t]{@{}p{6cm}@{}}--No effects found\end{tabular} \\ \hline
\textbf{\makecell{Syllable 2\\ Stressed}} &
  \begin{tabular}[t]{@{}p{5cm}@{}}--No effects found\end{tabular} &
  \begin{tabular}[t]{@{}p{5cm}@{}}-- Greater looks to words with higher pitch given higher risetime d$'$ sensitivity\\ -- Greater looks given higher Italian LexITA\\ -- Greater looks to words with longer duration given higher formant d$'$ sensitivity\end{tabular} \\ \hline
\end{tabular*}
\end{table}

\subsection{Individual differences in Italian word recognition}

\subsubsection{Auditory sensitivity contributed to variable eye movements}
\hl{Our exploratory study highlighted how stress cues are used variably across individuals. We found variable processing strategies that not only depend on the cue sensitivity of the individual but also on the specific multi-dimensional cues for each word. Recall that our replication results showed that a higher spectral tilt on the first syllable---which is informative for antepenultimate stressed words given the decrease of spectral tilt in the second syllable---resulted in greater looks to the antepenultimate stressed word. We found that individuals with less risetime sensitivity showed fewer looks to the antepenultimate targets given a higher spectral tilt. In other words, while our replication showed an overall effect of spectral tilt, this appears to have been driven by individuals with a certain sensitivity to risetime information in the signal. We also found evidence that pitch sensitivity and duration sensitivity can respectively influence eye movements towards (higher pitch sensitivity) or away from (lower duration sensitivity) the antepenultimate stressed word during the first syllable. We note that our additional exploratory interaction model} (see Figure \ref{fig:analysis_3_plot} right side) \hl{revealed an interaction between pitch and duration. Given that previous research has shown that f0 and duration are perceptually more salient than spectral cues or intensity} (see \cite{ip2022search, beckman1986stress, wright2009review, beckman_1994}), \hl{it seems likely that these two cues and their interaction play a role during stress perception of the first syllable. Future research will need to carefully tease these cues apart using different stimuli, participants, and tasks.}
 
\hl{In our analyses of the second syllable, we found a consistent pattern for antepenultimate stressed words: a higher pitch on the second syllable resulted in fewer looks to the antepenultimate stressed targets. Individuals with greater risetime sensitivity also showed fewer looks to the antepenultimate stressed target. This aligns with our overall replication finding, as well. We believe this is new evidence for the role of f0 height as an informative stress cue, particularly for reducing activation of the antepenultimate competitor and thereby increasing activation of the penultimate target, the latter of which should have a higher f0 on both syllables' vowels.}

\hl{Whereas we found null results for our penultimate syllable one analysis, we found two interesting results in our penultimate syllable two analysis. First, we found greater looks to words with higher pitch for individuals with higher risetime sensitivity. As we previously noted, penultimate stressed words have higher pitches on both vowels, which makes f0 height an informative cue. It is odd that risetime sensitivity was a significant predictor here and not pitch sensitivity. We note that our additional exploratory interaction model} (see Figure \ref{fig:analysis_3_plot} right side) \hl{revealed an interaction between pitch and amplitude for the second syllable. Future work will need to clarify whether this finding was a result of our stimuli, participants, or task. 

Second, we found greater looks to words with longer durations for individuals with higher formant sensitivity. Penultimate stressed words have longer second syllable durations, so duration is an informative cue. Our exploratory interaction model revealed interactions between duration and pitch and between duration and amplitude. This pattern seems to indicate the relative inseparability of stress from segmental information in Italian} \citep{CutlerNorris1988, Tagliapietra2005}.

\hl{Thus, we believe spoken Italian word recognition depends not only on the acoustic cues, but also the listener's auditory sensitivity. This evidence supports claims that speech perception involves the optimal integration of acoustic cues} \citep{clayards2008} and that some listeners are better than others at making use of multiple cues \citep{clayards2018}. \hl{Our results suggest a time-course in which spectral tilt and f0 height are the most informative cues during the first syllable. Being more sensitive to these cues can lead to reduced competition and earlier word recognition, particularly for antepenultimate stressed words. During the second syllable, f0 height continues to play a role as does duration} (cf. \citep{Alfano2006,Alfano2009}. \hl{Importantly, this does not imply that amplitude plays no role in the detection of stress} (see \citep{Maturi1998, Sulpizio_McQueen_2012}), \hl{but rather it interacts with another cue as we found in our exploratory interaction models.}


\subsubsection{Language proficiency in the L1 and L2 affect Italian word prediction}

We found a clear and exciting asymmetry in our results driven by L1 and L2 lexical knowledge. Participants with higher L1 Italian LexITA scores looked to the (stressed) penultimate second syllable at a greater rate. One interpretation of these results is that listeners with a sufficient Italian lexicon store abstract knowledge about Italian stress patterns and use that information only during the second syllable after the acoustic information from the first syllable is processed. Remember that \cite{Sulpizio_McQueen_2012} \hl{tested university students, who may have had a relatively large L1 Italian lexicon given their education. This population of listeners may have inflated their results. We only found this pattern of eye-fixations once we considered individual differences. Although we did not ask our participants about their educational background, there does seem to be a small group of participants with a higher LexITA score, almost resulting in a bimodal distribution} (see Figure \ref{fig:plot_raw_task}, second plot from right). \hl{It is very possible that these participants were similar to those tested by} \cite{Sulpizio_McQueen_2012}. \hl{We refrain from speculating beyond this given we lack our participants' educational background, but this potential finding highlights the importance of testing a wider population and always asking participants about their education (we thank the anonymous reviewer for this important reminder).}

Interestingly, we also observed that participants with higher L2 English LexTALE scores looked to the antepenultimate target at a greater rate. We interpret this pattern as evidence of the L2 affecting the L1 \citep{marian2003competing, dijkstra2002architecture}. English may be helpful in recognizing antepenultimate word recognition due to the higher frequency of first syllable stress in English \cite{cutler2007dutch}. That is, those with greater English skills are co-activating more first syllable stressed words. Surprisingly, we did not find evidence that there were any negative effects of higher language proficiency on word recognition. This is in contrast with \cite{primativo2013bilingual}'s reading study. One explanation for these conflicting results is that while production may be somewhat hindered by cross-linguistic differences in stress cues, perception benefits from it.

Taken together, these novel results show that both L1 and L2 lexical knowledge can affect the real-time prediction of spoken Italian words. This finding is in line with reading research which has shown larger lexical knowledge results in greater statistical knowledge \citep{mirman2008attractor, kuperman2013reassessing}.


\subsubsection{Working memory and autism spectrum quotient do not predict eye movements}
To our surprise, we found null effects of both working memory and autism spectrum quotient in all our exploratory models. This was unexpected given the considerable number of studies that have found working memory as a reliable predictor \citep{Traxler2009, Huettig2016}. Both of our measures appeared to be normally distributed (see Figure \ref{fig:plot_raw_task}), indicating we had a wide range of participants and behavior. Both tasks were fairly reliable as measured by Cronbach's alpha. These null results may be due to a number of reasons. \hl{For working memory, there could be a different involvement of working memory in the prediction of sentence continuations and in the prediction in spoken word recognition} (see \cite{huettig2022parallel} \hl{for discussion on between-item and within-item prediction). Similarly, either variability in working memory is not as useful for prosodic processing over only two syllables (floor effect) or our sample was still not varied enough in working memory---or large enough---to capture differences.} 

For AQ, the fairly straightforward design of our study may have been too simple to observe robust effects of autism spectrum quotient (cf. \cite{Sinagra2022} which tested prosody detecting involving face masks that obscured facial cues). Moreover, we took into account acoustic cue sensitivity and language proficiency. The effects found in other studies for prosody deficiency for autism characteristics may be the result of not controlling for more basic cognitive measures \cite{grossman2023relationship}. \hl{Another possibility is that we may not have captured data from any participants that actually show a high tendency for autistic traits. Of all our participants, only two scored above the critical minimum necessary .64 (32 from }\cite{Baron-Cohen2001}).\hl{ That is, while our sample was larger, only approximately 4.2\% scored above the critical minimum meaning that our sample may not contain enough variability to drive an effect. Considering that ASD is only diagnosed in about 1\% of the general population} \citep{who2023autism}\hl{, we would need a much larger sample to capture possible effects.}


\subsubsection{On psycholinguistics replications}
A key motivation for this study was to contribute to the growing need for close replication \citep{Marsden_2018}. Our results demonstrate the success (and relative ease) of carrying out a web-based eye-tracking study with a larger, more heterogeneous population of participants \citep{Vos_2017, bramlett_wiener_24-AOW}. However, as \cite{mcmanus2022replication} notes, \hl{even minor changes during replication can lead to significant constraints. We encountered three such challenges: the use of multiple tasks and the shift to an online environment.

First, in our attempt to use pre-existing materials to maintain similarity across studies, we "cloned" the LexTALE using Gorilla platform’s open materials. The version imposed a built-in time limit on the task. However, as the editor pointed out, both the LexTALE and LexITA do not have time limits in their original forms, which makes the scores from our timed version incomparable to those of the untimed standard version due to the introduction of a speed-accuracy trade-off.

Second, the shift to an online study, while intended as the only change, resulted in a cascade of unintentional modifications. One significant consequence was a difference in participant population. This shift has been observed in other web-based replications as well} \citep{bramlett_wiener_24-AOW}, \hl{where changes in study logistics inadvertently influence outcomes in ways that diverge from the original study design.

Third, while attempting to replicate the analyses of the original study, we encountered difficulties in fully understanding how the original data wrangling and modeling were fully conducted. Although we aimed to remain as close as possible to the original procedures while maintaining statistical rigor, these two goals were sometimes in conflict. As a result, certain analytical choices may have diverged from the original study, reflecting both the need for clarity in replication and open practices, as well as the challenge of balancing methodological fidelity with sound statistical practices.}

\subsubsection{Limitations}

Despite a larger and more diverse sample size than \cite{Sulpizio_McQueen_2012}, our participant pool was still limited to Italian speakers recruited online, potentially introducing selection bias. Conducting the study online also presented inherent challenges in controlling the experimental environment. While participants completed the tasks in various environment that may be more naturally varied than a lab setting, i.e., noise levels and device quality might have affected the results (see \cite{bramlett_wiener_24-AOW} for discussion), however with rigorous screening, attention checks, and reliability test we believe that our results are reliable.

\hl{Our statistical analyses, while more conservative and nuanced in controlling for colinearity, introduced complexity that complicates direct comparisons with the original study.} Additionally, our findings on individual differences, particularly the null effects for working memory and autism spectrum quotient, suggest an interconnected relationship between cue sensitivity, language proficiency (in the L1 and L2), working memory, and AQ. \hl{(An additional possibility in the null effects could be due to floor effects).}  Our online R code shares further exploratory correlation analyses for interested readers. Future research should consider a broader array of individual difference measures, including cognitive flexibility and attention control, to provide a more comprehensive picture. While our sample (n=47) was larger than \cite{Sulpizio_McQueen_2012}'s \hl{sample (n=32), which should mean that our replication analyses are sufficiently powered, we did not run power analyses, which may mean that some effects not found in the extension analysis could be due to low power (e.g., AQ)}.

Furthermore, focusing exclusively on Italian limits the generalizability of our findings to other languages. Cross-linguistic studies are needed to determine the universality of our findings, as stress patterns and their acoustic correlates vary significantly across languages \cite{cutler2007dutch}. The cross-sectional nature of our study also precludes conclusions about the developmental trajectory of stress processing abilities. Longitudinal studies tracking individuals over time could shed light on the stability and consistent use of stress cues for individuals in different environments. Future studies should explore a wider range of individual difference measures and employ longitudinal designs to assess how these factors interact over time. Examining other languages and specialized populations, as well as exploring how continuum-manipulated speech contrasts affect Italian word recognition, would provide additional insights into the cues most useful for word recognition.

%Finally, The application of k-nearest neighbors (k-NN) and random forest models provided further insights, confirming the significance of pitch and duration in predicting target fixations to stress patterns across both syllables. These machine learning approaches reinforced our parametric findings and demonstrated the robustness of our analytical framework. Our study contributes to the growing body of literature on individual differences in language processing by demonstrating how lexical proficiency and auditory processing skills modulate the integration of lexical stress cues. Unlike prior research, working memory did not come up significant in the parametric models. However, our k-NN and random forest analysis indicate that working memory is an important modulator during word recognition. Similarly, ASDQ scores also showed up as an important variable in these models. These results indicate a need for examining non-linear relationships between individual differences and cues during word recognition in order to fully understand how cognitive abilities affect the time course of lexical processing. The lack of findings for working memory in parametric models could also be due to the fact that we are looking at low frequency words at the level of word recognition rather than sentence processing. These findings have important implications for models of spoken word recognition, suggesting that cognitive and linguistic factors play a central role in word recognition and that various strategies are employed by different listeners for each word.

