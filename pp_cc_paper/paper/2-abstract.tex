This study successfully replicates and extends \cite{Sulpizio_McQueen_2012} by examining the role of lexical stress in word recognition among Italian speakers, incorporating a diverse web-based sample and a broad range of individual difference measures. Participants completed tasks measuring working memory, lexical proficiency in English and Italian, autism spectrum characteristics, and sensitivity to speech cues, alongside an eye-tracking experiment using the visual world paradigm to examine tri-syllabic Italian words with either penultimate or antepenultimate stress. We used Generalized Linear Mixed Effects Models (GLMMs), LASSO regression, and k-Nearest Neighbors (k-NN), to explore cue integration and fixation proportion across Italian stress patterns. Results indicate that higher lexical proficiency and auditory processing skills enhance word recognition, with sensitivity to duration and pitch cues modulating fixation patterns based on stress type. Models of Individual differences and acoustic cues emphasize the complexity of spoken language processing. Our findings suggest that listeners use stress information as early as the first syllable. Dissimilar to \cite{Sulpizio_McQueen_2012}, we found that stress cues are integrated in both penultimate and antepenultimate words. These findings highlight the significance of cognitive and linguistic factors in word recognition models and suggest diverse listener strategies. Future research should continue to explore these factors to further understand language processing variability.


This study's goals are twofold

1) to replicate an in-person eye-tracking study, using the internet and web cameras; 2) to extend the study by examining a variety of individual difference predictors. These behavioral measures are used to model the captured eye-movements. We focus on Italian lexical stress

In this study, we aim to address this gap by investigating the role of individual differences in language prediction using a more heterogeneous web-based sample. We replicate and extend the work of \cite{Sulpizio_McQueen_2012}, who explored the effects of lexical stress on word recognition, by incorporating a broader range of individual difference measures, including working memory, lexical proficiency, autism spectrum quotient, and speech cue sensitivity (i.e., pitch, format, duration, and risetime). Our approach employs modern statistical techniques such as Generalized Linear Mixed Effects Models (GLMMs) and their linear counterparts (GLMM), LASSO (Least Absolute Shrinkage and Selection Operator) regression, and k-Nearest Neighbors (k-NN) to model these effects more effectively and interpretably.