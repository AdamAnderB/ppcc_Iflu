This study replicates and extends \cite{Sulpizio_McQueen_2012}'s Experiment 1, which examined how acoustic information and abstract knowledge about lexical stress are used during Italian spoken word recognition. This visual world paradigm study simultaneously presented tri-syllabic Italian words visually and \hl{auditorily.} The target and competitor overlapped segmentally during the first two syllables but differed in penultimate or antepenultimate stress. We replicate their in-person eye-tracking study by using webcams to test a larger and more diverse sample (N = 47). Our findings corroborate the original study by demonstrating listeners use stress information as early as the first syllable and make use of abstract knowledge about Italian stress patterns. We extended their study by testing participants' working memory, lexical proficiency in English and Italian, autism spectrum quotient, and sensitivity to pitch, duration, risetime, and formants. Dissimilar to \cite{Sulpizio_McQueen_2012}, we found that stress cues are integrated for recognition of both penultimate and antepenultimate words and that individuals vary in their reliance on cues and the timing of cue integration. These exploratory findings highlight the role of cognitive and linguistic factors in word recognition and suggest diverse listener strategies in processing and predicting prosody. 