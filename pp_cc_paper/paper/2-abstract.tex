This study closely replicates and extends \cite{Sulpizio_McQueen_2012}'s Experiment 1, which examined how acoustic information and abstract knowledge about lexical stress are used during Italian spoken word recognition. This visual world paradigm study simultaneously presented trisyllabic Italian words visually and auditorily. The target and competitor overlapped segmentally during the first two syllables but differed in penultimate or antepenultimate stress. We replicate the original in-person eye-tracking study by using webcams to test a larger and more diverse sample (N = 47). Our findings corroborate the original study by demonstrating listeners use stress information as early as the first syllable to recognize spoken words. We found that first syllable spectral tilt and f0 information are predictive of eye-movements to antepenultimate stressed targets. We did not find overall evidence of listeners using abstract knowledge to recognize penultimate stressed words. We further extended the original study by testing participants' individual differences in auditory sensitivity to pitch, duration, risetime, and formants, along with their working memory, lexical proficiency in English and Italian, and autism spectrum quotient. Our exploratory results showed that individuals vary in their reliance on cues and the timing of cue integration, that stress cues are integrated for recognition of both penultimate and antepenultimate stressed words, and that L1 Italian and L2 English lexical knowledge can affect eye-movements. 